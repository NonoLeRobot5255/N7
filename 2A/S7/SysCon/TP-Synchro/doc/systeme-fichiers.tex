%===============================================================================
%
%===============================================================================
\section{Introduction}

   Nous allons décrire ici les outils permettant de définir et
manipuler les fichiers ainsi que les interfaces permettant d'accéder à
des fichiers.

   Nous verrons également que ces interfaces permettent l'accès à d'autres outils
d'échange d'information. C'est en effet une pratique classique que de
proposer une interface la plus simple et la plus homogène possible à
tous les outils de stockage et d'échange de données.

   Nous allons commencer par essayer de définir clairement les deux
principales structures de données classiquement utilisées : le {\em
  noeud d'index} et le {\em fichier ouvert}.

   La première de ces structures donne des informations sur la
structure et les propriétés du fichier ; c'est au travers d'elle que
nous pourrons savoir où se trouve physiquement le fichier, à qui il
appartient, qu'est-ce qu'un utilisateur a le droit d'en faire, etc
\ldots

   La seconde contient des informations plus dynamiques sur un fichier
qui est manipulé par une tâche. On y trouvera donc par exemple la
position actuelle dans le fichier.

   Une analogie simple nous permet de dire que le n\oe{}ud d'index
nous permet de savoir qu'un livre est situé dans la troisième rangée
de la deuxième étagère (et qu'il est interdit d'écrire sur le livre
!), la où le fichier ouvert contient par exemple un marque page
positionné à la prochaine page qu'un utilisateur s'apprète à lire.

%===============================================================================
%
%===============================================================================
\section{Notion de noeud d'index}

   La structure permettant de décrire les propriétés d'un fichier est
classiquement appelée {\em n\oe{}ud d'index}, ou plus souvent {\em
  inode} (pour {\em index node}).

%-------------------------------------------------------------------------------
%
%-------------------------------------------------------------------------------
\subsection{Dans ManuX}

   ManuX définit la structure \lstinline!NoeudI! et les fonctions
associées  dans le fichier \fichier{include/manux}{noeudi}{h}. Les
fonctions sont implantées dans le fichier \fichier{sf}{noeudi}{c}.

   Une chose importante à avoir en tête est le fait qu'un fichier
décrit par un NoeudI peut être de diverses natures et qu'il est donc
important de garder trace d'informations permettant de savoir comment
le manipuler.

   La structure \lstinline!NoeudI! intègre en particulier les champs
suivants~:

\begin{itemize}
   \item \lstinline!methodesFichier! fourni les fonctions de
     manipulation d'un fichier du type concerné.
   \item \lstinline!prive! est un pointeur vers des données
     spécifiques au type du fichier et permettant de définir des
     propriétés spécifiques du fichier.
\end{itemize}
    
\subsubsection{Les méthodes de manipulation des fichiers}

%===============================================================================
%
%===============================================================================
\section{Description d'un fichier ouvert}

%-------------------------------------------------------------------------------
%
%-------------------------------------------------------------------------------
\subsection{Dans ManuX}

   Dans ManuX, un fichier ouvert est décrit dans la structure
\lstinline!fichier! qui contient en particulier les données suivantes
   
\begin{itemize}
   \item \lstinline!noeudI! 
   \item \lstinline!prive! est un pointeur vers des données
     spécifiques au type du fichier et permettant de conserver trace
     de l'état du fichier.
\end{itemize}
   

%===============================================================================
%
%===============================================================================
\section{Outils de base de manipulation des fichiers}

%-------------------------------------------------------------------------------
%
%-------------------------------------------------------------------------------
\subsection{Interface utilisateur}

   L'interface fournie à l'espace utilisateur est généralement simple
et la plus abstraite possible. Elle passe par un certain nombre
d'appels système permettant en particulier d'ouvrir, de lire, d'écrire
et de fermer un fichier.

%-------------------------------------------------------------------------------
%
%-------------------------------------------------------------------------------
\subsection{Dans le noyau}

%-------------------------------------------------------------------------------
%
%-------------------------------------------------------------------------------
\subsection{Dans ManuX}

   ManuX définit une interface utilisateur classique fondée sur
quelques appels systèmes inspirés de ce que l'on trouve dans un
système Unix.

%...............................................................................
%
%...............................................................................
\subsubsection{L'interface}

%
%
%
\paragraph{Ouverture d'un fichier}

L'ouverture d'un fichier se fait à l'aide de l'appel système
\lstinline!UNDEFINED!.

%
%
%
\paragraph{Lecture dans un fichier}

L'appel système \lstinline!lire! est défini dans le fichier
\fichier{usr/include}{stdio}{h} de la façon suivante~:

\begin{lstlistings}
int lire(int fd, void * buffer, int nb);
\end{lstlistings}

%
%
%
\paragraph{Écriture dans un fichier}

%...............................................................................
%
%...............................................................................
\subsubsection{Dans le noyau}

%
%
%
\paragraph{Lecture dans un fichier}

   L'implantation de l'appel système de lecture est réalisée dans la
fonction \lstinline!sys_lire()! implantée dans le fichier
\fichier{sf/fichier}{c}. L'appel système est défini dans la fonction
\lstinline!sfInitialiser() du même fichier.

   Cette fonction peut servir en particulier à mettre en place des
contrôles de sécurité. Elle se contente pour le moment de faire appel
à la fonction \lstinline!fichierLire()! qui est en charge d'aiguiller
vers la fcontion de lecture spécifique au type du fichier ciblé.

%===============================================================================
%
%===============================================================================
\section{Un exemple : les consoles}

   Je ne vais pas revenir ici sur l'implantation des consoles, déjà
décrite avec brio dans le chapitre qui leur est dédié. Regardons
simplement comment elles se retrouvent affublées d'une interface de
type fichier.

   Dans le fichier \fichier{lib}{console}{c}, les fonctions de
manipulation d'une console sont définies de la façon suivante

\begin{lstinline}
/**
 * @brief Déclaration des méthodes permettant de traiter une console
 * comme un fichier
 */
MethodesFichier consoleMethodesFichier = {
   .ouvrir = consoleOuvrir,
   .ecrire = consoleEcrire,
   .lire = consoleFichierLire
};
\end{lstinline}

%-------------------------------------------------------------------------------
%
%-------------------------------------------------------------------------------
\subsection{La fonction d'écriture}

   La fonction \lstinline!consoleEcrire()! est alors simplement
constituée d'un appel à la fonction d'affichage dans la console visée,
dont un pointeur est obtenu via le champ privé.

   ATTENTION ! il faut aller le chercher via le noeudI, ce qui n'est
   pas le cas pour le moment !!!
   
%===============================================================================
%
%===============================================================================
\section{Les tubes de communication}

Les méthodes de manipulation
