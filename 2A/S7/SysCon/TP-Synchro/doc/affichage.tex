%===============================================================================
%      L'affichage
%===============================================================================
\section{Introduction}

   Une des premières choses à faire, afin de nous simplifier la vie
pour la suite du développement, est de se doter d'un service minimal
d'affichage. Nous allons donc mettre en place quelques outils très
simples fondés sur ce que nous founit le matériel.

   Nous ne nous intéresserons ici qu'à l'affichage des messages du
noyau, pas de ceux des éventuels processus utilisateurs. Nous
étendrons cependant un peu ces outils de sorte à nous permettre cela
par la suite.

   Nous allons procéder ici en trois étapes principales.

\begin{itemize}
   \item   Dans un premier temps, nous allons écrire le code
permettant un accès simple à l'écran et une gestion minimale de ce
dernier dans un module ``{\em Console}''. Cet outil est suffisant pour
que notre noyau soit capable de nous donner un signe de vie à
l'écran. Mais nous n'allons pas nous arrêter en si bon chemin !

   \item Nous pourrons alors écrire une fonction \lstinline!printk!
utilisable partout dans le noyau et chargée de mettre en forme des
messages riches et de les afficher sur la console,  \ldots{} ou ailleurs.

   \item   Nous construirons là-dessus un ``journal'' du noyau, c'est-à-dire
un outil permettant de rassembler et d'organiser les messages émis par
les différents sous-systèmes du noyau au travers de {\tt printk}.
Ils pourront être affichés sur la console, ou conservés pour être
stockés/traités d'une autre façon.
   Nous modifierons alors la fonction {\tt printk} de sorte à ce qu'elle
envoie au journal ce qui doit être affiché. C'est lui qui aiguillera
vers la bonne destination.

\end{itemize}

   L'avantage d'une telle structure est que nous pourrons la faire
évoluer. Nous pourrons par exemple créér des consoles virtuelles
fournissant le même service que notre outil de gestion de l'écran. Il
sera très simple de les intégrer dans le noyau. De même, nous pourrons
définir un affichage avec plusieurs niveaux d'importance des messages
qui pourront ensuite être aiguillés en fonction de leur criticité.

%-------------------------------------------------------------------------------
%
%-------------------------------------------------------------------------------
\section{Gestion de l'écran : la console}

   Les fichiers \fichier{include/manux}{console}{h} 
et \fichier{lib}{console}{c} définissent et implantent un outil de
gestion de base de la console.

   Nous allons pour le moment initialiser et utiliser une console
unique, que nous appellerons la ``console noyau''. Tous les messages
émis par le noyau y seront transmis.

   Nous allons définir essentiellement les éléments suivants

\begin{itemize}
   \item Le type \lstinline!Console! décrit ce qu'est une console.
   \item La fonction \lstinline!consoleInitialisation()! initialise le
     système de gestion de la console. Elle crée en particulier la
     console noyau. Cette fonction est invoquée une fois et une seule
     lors du démarrage du noyau.
   \item La fonction \lstinline!consoleAfficher()! permet d'afficher
     une chaîne de caractères sur la console.
\end{itemize}  

   Quelques autres fonctions, de moindre importance, sont également
définies.
   
%...............................................................................
%
%...............................................................................
\subsection{Fonctionnement de base}

   Il se trouve que l'écran d'un PC peut être utilisé en mode textuel
comme un simple tableau de caractères. Ce tableau est situé à une
adresse bien connue $a$ et contient $l$ lignes de $c$
caractères. Chaque caractère est codé sur deux octets : l'un définit
le code {\sc ascii} du caractère affiché et l'autre la couleur du
texte et du fond. 

%...............................................................................
%
%...............................................................................
\subsection{Définition du type {\tt Console}}

   Le type permettant de définir une console est défini dans le fichier
\fichier{include/manux}{console}{h}. Pour le moment nous ne créerons
qu'une console, mais lorsque nous allons introduire les consoles
virtuelles, nous étendrons ce type.

   Le type \href{html/console_8h_source.html#l00062}{\tt Console} est
donc très simple, il comprend quelques champs décrivant les
caractéristiques de l'écran (sa taille, l'adresse de la zone mémoire)
et l'état actuel de l'affichage (position du curseur et de
l'attribut).
   
%...............................................................................
%
%...............................................................................
\subsection{Initialisation de la console}

   L'initialisation du système de gestion de console (assurée par la
fonction \lstinline!consoleInitialisation()!) consiste essentiellement
en l'initialisation de la console noyau.

   La fonction d'initalisation de la console, implantée dans
\fichier{lib}{console}{c}, est elle aussi très simple. Elle se
contente d'une initialisation à des valeurs par défaut des champs du
type {\tt Console}.

   Nous n'utilisons ici pour le moment qu'une seule console, elle sera
définie comme une variable statique du fichier
\fichier{lib}{console}{c} et pourra être obtenue par la fonction {\tt
  consoleNoyau()} lorsque n'écessaire.

   En revanche, en prévision de la création de consoles virtuelles,
toutes les fonctions d'affichage prennent en première paramètre un
pointeur sur la console.

%...............................................................................
%
%...............................................................................
\subsection{Affichage sur la console}

   On construit ici en particulier la fonction suivante, chargée
d'afficher sur la console une chaîne de caractères passée en paramètre~:

\begin{lstlisting}
void afficherConsole(Console * cons, char * msg);
\end{lstlisting}

   Elle prend en paramètre une chaîne de caractères ``classique''
(terminée par un caractère nul) et l'affiche sur la console sans aucune
mise en forme.

   La gestion de la console est une des toutes premières choses
initialisées par le noyau, de sorte à pouvoir faire coucou dès que
possible.

%-------------------------------------------------------------------------------
%
%-------------------------------------------------------------------------------
\section{Un premier {\tt printk()}}

   Grâce à notre console, l'écriture de {\tt printk()} est assez
simple, puisque cette fonction n'a plus qu'à se préoccuper de créer
une chaîne de caractères et de la transmettre au journal (ou
directement à la console si on n'utilise pas le journal).

   Elle est définie dans le fichier \fichier{include/manux}{printk}{h}
   et mise en \oe{}uvre dans le fichier
   \fichier{noyau}{printk}{c}

   La fonction essentielle est bien sûr :

\begin{lstlisting}
   void printk(char * format, ...);
\end{lstlisting} 

   Dont le rôle est de mettre en forme une chaîne de caractères
qu'elle envoie ensuite à la console (ou au journal lorsque nous
l'aurons créé). Elle utilise pour cela le code suivant

\begin{lstlisting}
   afficherConsoleN(consoleNoyau(), chaine, indice);  
\end{lstlisting}

   Le paramètre \lstinline!indice! donne la longueur de la chaîne de
caractères \lstinline!chaine! qui est ainsi affichée sur la console du
noyau.
   
%-------------------------------------------------------------------------------
%
%-------------------------------------------------------------------------------
\section{Mise en place d'un journal}

   Le journal est spécifié dans le fichier
\fichier{include/manux}{journal}{h} et implanté dans le fichier
\fichier{lib}{journal}{c}.
   
   Nous allons définir un mécanisme de journal permettant de conserver
les messages transmis par le noyau. Pour le moment, notre journal se
contente de transmettre ces messages sur l'écran physique. Lorsque
nous aurons mis en place des outils de gestion mémoire, nous pourrons
étoffer ce mécanisme de journal.

   Le journal est implanté dans les fichiers
\lstinline!manux/journal.h! et \lstinline!lib/journal.c!. Il
propose essentiellement la fonction suivante

\begin{lstlisting}
   void journaliser(char * message); 
\end{lstlisting}

   Elle utilise la fonction \lstinline!afficherConsoleN()! pour
envoyer le message à la console.

   La fonction \lstinline!printk()! est alors modifiée pour utiliser la
fonction \lstinline!journaliser()! plutôt que passer directement par
la console.

   Évidemment, le journal est initialisé juste après la console.

%-------------------------------------------------------------------------------
%
%-------------------------------------------------------------------------------
\subsection{Introduction de plusieurs niveaux de messages}

   Maintenant que nous avons un système qui permet à tous les
sous-systèmes d'envoyer des messages, essentiellement {\em via} la
fonction \lstinline!printk()!, nous allons mettre en place un système
permettra de filtrer un peu tous ces messages.

   Pour cela, l'idée est d'utiliser la fonction \lstinline!printk()!
de la façon suivante

\begin{lstlisting}
   printk("{2} Un message\n");
\end{lstlisting}

   Ce sont les trois premiers caractères qui vont permettre de
réaliser cet aiguillage.

   C'est plus précisément le chiffre qui va permettre de définir le
niveau de criticité du message. Les valeurs suivantes sont définies
dans \fichier{include/manux}{journal}{h}~:

\begin{lstlisting}
#define MANUX_JOURNAL_NIVEAU_PANIQUE      0
#define MANUX_JOURNAL_NIVEAU_URGENCE      1
#define MANUX_JOURNAL_NIVEAU_CRITIQUE     2
#define MANUX_JOURNAL_NIVEAU_ERREUR       3
#define MANUX_JOURNAL_NIVEAU_ATTENTION    4
#define MANUX_JOURNAL_NIVEAU_NOTIFICATION 5
#define MANUX_JOURNAL_NIVEAU_INFORMATION  6
#define MANUX_JOURNAL_NIVEAU_DEBUGAGE     7
\end{lstlisting}  

   Les caractères qui encadrent le niveau de criticité permettent de
définir le comportement. Précisons ici en effet que, dès que nous
aurons mis en place des outils le permettant, nous allons associer un
fichier au journal. Ce fichier permettra par exemple de conserver les
messages du noyau, même après l'arrêt du système (ce qui est affiché
sur la console disparait, ce qui nest pas pratique pour une analyse a
posteriori).

La signification de ces caractères est alors la suivante~:
\begin{itemize}
   \item \lstinline![n]! et le message sera envoyé sur la console ;
   \item \lstinline!(n)! le message sera envoyé sur le fichier
     associé au journal ;
   \item \lstinline!{n}! le message sera transmis sur les deux
      destinations.     
\end{itemize}

   Le code de la fonction \lstinline!journaliser()! est donc modifier
en conséquence. Elle utiliser dans un premier temps la fonction
\lstinline!aiguillerMessage()! qui traite ce ``préfixe'' de sorte aue
le message soit transmis correctement, puis dans un second temps la
fonction \lstinline!journaliserNiveau()! qui transmet effectivement le
message à la (ou les) bonne(s) destination(s).

%-------------------------------------------------------------------------------
%
%-------------------------------------------------------------------------------
\section{Plus de souplesse avec les consoles virtuelles}

   Lorsque nous allons mettre en place la notion de tâches, il sera
intéressant d'attribuer à chacune d'entre elles une console virtuelle
potentiellement différente des autres. Nous pourrons ainsi basculer de
l'une à l'autre pour mieux distinguer ce qu'affiche chaque tâche.

%-------------------------------------------------------------------------------
%
%-------------------------------------------------------------------------------
\subsection{Utilisation des fichiers}

