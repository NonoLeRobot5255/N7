%===============================================================================
%      Les appels système
%===============================================================================
\section{Les appels systèmes}

   Les appels systèmes constituent une part importante de l'interface
entre le noyau et les applications. Dans ManuX, le choix a été fait
d'implanter ce mécanisme au travers d'une interruption. C'est une
technique classique (utilisée par exemple par Linux).

   Leur mise en \oe{}uvre passe par plusieurs étapes que je vais
détailler ici. L'objectif d'un appel système est d'offrir à une
application utilisateur un service qui ne peut être rendu que part le
noyau. Pour des raisons de confort d'utilisation, ces services sont
rendus sous la forme de fonctions classiques, comme une librairie
traditionnelle. Pour en arriver là, nous pouvons distinguer quatre
étapes

\begin{itemize}
   \item L'application utilise une fonction C pour invoquer un appel
     système. Le code de cette fonction doit donc être construit de
     sorte à préparer puis mettre en \oe{}uvre la sollicitation du
     service auprès du noyau (par le biais d'une interruption ici).
   \item L'interruption est déclanchée et le code du handler de cette
     interruption est exécuté.
   \item Le handler va devoir déterminer quelle est la fonction du
     noyau qui doit être utilisée pour rendre le service voulu à
     l'application.
   \item La fonction en question va être exécutée et le service fourni
     à l'application.
\end{itemize}

   Nous allons maintenant décrire la mise en \oe{}uvre de ces étapes
en prenant comme exemple un appel système qui nous permettra d'envoyer
un message sur la console. Nous pourrons ainsi écrire un
\lstinline!printf! pour les applications !

%-------------------------------------------------------------------------------
%      Définition de l'interface
%-------------------------------------------------------------------------------
\subsection{Définition des éléments communs noyau/application}

   Le fichier \fichier{manux}{appelsystemenum}{h} donne en particulier les
numéros des différents appels systèmes. Il doit donc absolument être
commun au noyau et aux ``applications'' (voir la documentation sur
l'espace utilisateur).

%-------------------------------------------------------------------------------
%      Définition de l'interface
%-------------------------------------------------------------------------------
\subsection{Définition de l'interface}

   Ce que j'appelle interface est la signature de la fonction
directement utilisable dans une application pour invoquer un appel
système.

   Cette interface est donc décrite côté applications, dans les
fichiers de {\tt usr}.
   
   Des macros ont été créées dans le fichier \fichier{usr/include/manux}{appelsysteme}{h}
pour simplifier une telle définition. Ce
sont par exemple les ``fonctions'' \lstinline!appelSysteme0!,
\lstinline!appelSysteme1!, \ldots 

   On utilisera donc une telle macro chaque fois qu'il sera nécessaire
de définir un nouvel appel système. C'est ainsi que dans le fichier
\lstinline!usr/include/unistd.h! on trouve par exemple :

\begin{lstlisting}
appelSysteme2(NBAS_AFFICHER, int, afficherConsole, void *, int);
\end{lstlisting}

   Cette ligne défini donc la fonction \lstinline!afficherConsole()!
qui va permettre à une application d'envoyer un message sur la console.

%-------------------------------------------------------------------------------
%      Ecriture de l'appel système
%-------------------------------------------------------------------------------
\subsection{Écriture de l'appel système}

   La partie la plus importante est naturellement d'écrire le code de
l'appel système. Celui-ci consiste simplement en une fonction C
classique (qui en utilise éventuellement d'autres, naturellement). La
seule contrainte étant que si cette fonction a des paramètres, le
premier d'entre eux (qui ne présente aucun intéret pour le
programmeur) doit être de type \lstinline!ParametreAS!.

   Ce paramètre sert uniquement a ``recadrer'' les paramètres suivants 
dans la pile suite aux décalages induits dans celle-là par les appels
successifs à l'interface de l'appel système, à l'interruption, et à
l'appel système lui-même d'une part, et à la sauvegarde des registres
d'autre part.

   Observons par exemple le code permettant la mise en \oe{}uve de
l'appel système \lstinline!ecrireConsole!. Il se trouve dans le
fichier {\tt lib/console.c} :

\begin{lstlisting}
/**
 * L'implantation de l'AS ecrireConsole
 */
int sys_ecrireConsole(ParametreAS as, void * msg, int n)
{
   assert(tacheEnCours != NULL);
   Console * cons = tacheEnCours->console;
   
   assert(cons != NULL);
   afficherConsoleN(cons, msg, n);

   return n; 
}
\end{lstlisting}

   Bien sûr, cette fonction utilise le code de manipulation de la
console déjà écrit.   

%-------------------------------------------------------------------------------
%      Enregistrement de l'appel système
%-------------------------------------------------------------------------------
\subsection{Enregistrement de l'appel système}

   L'enregistrement de l'appel système dans le système d'exploitation est
réalisé par une invocation de la fonction suivante, définie dans {\tt
appelsysteme.h} et codée dans {\tt noyau/appelsysteme.c} :

\begin{lstlisting}{}
int definirAppelSysteme(int num, void * appel);
\end{lstlisting}

   où 

\begin{description}
   \item[\lstinline!num!] est le numéro de l'appel système enregistré ;
   \item[\lstinline!appel!] est la fonction de traitement de cet appel
système.
\end{description}

   Le travail de cette fonction est donc simplement de placer dans la
table des appels systèmes (\lstinline!vecteurAppelsSysteme!) la
fonction de traitement de cet appel système.

   C'est donc ici la ``partie noyau'' qui est réalisée, c'est-à-dire
la définition des actions à mener lorsque l'appel système est invoqué.

   Nous trouvons par exemple dans la fonction
\lstinline!initialiserAppelsSysteme()! du fichier {\tt
  noyau/appelsysteme.c} la ligne de déclaration de notre appel système
\lstinline!ecrireConsole! :

\begin{lstlisting}{}
void initialiserAppelsSysteme()
{
   ...
  
   /* Envoyer une chaine de caracteres sur la console */
   definirAppelSysteme(NBAS_ECRIRE_CONS, sys_ecrireConsole);

   ...
}
\end{lstlisting}{}

