%===============================================================================
%
%===============================================================================
\section{Introduction}

\subsection{Préambule}
   Ce document n'est constitué pour le moment que de quelques notes
que j'ai pu prendre en essayant de réaliser le début d'un truc qui pourrait
vaguement laisser penser que j'essaye de réaliser un noyau de système
d'exploitation. 

   Ces notes me sont purement personelles et je vous invite donc, qui que
vous soyez, arrêtez de lire ça ! Vous êtes encore là ? C'est pas sympa \ldots

%-------------------------------------------------------------------------------
%
%-------------------------------------------------------------------------------
\subsection{Présentation}

   Ce document rassemble quelques notes prises durant ma tentative de
dévelop\-pement de \manux (``Misérable Approximation de Noyau
Un*X''). Mon but en développant cette chose est uniquement de mieux
comprendre les problèmes liés à l'implantation d'un système
d'exploitation. Je n'ai en aucun cas comme objectif d'arriver à un
{\sc os} utilisable à d'autres fins que tester des mécanismes
systèmes. 

   De plus, je ne recherche absolument pas la performance dans le code 
que j'écris, mais plutôt la lisibilité et la compréhensibilité. Ça
peut ne pas sauter aux yeux dans les parties de code encore en
développement (c'est à dire partout pour le moment) \ldots 

%===============================================================================
%       Organisation générale
%===============================================================================
\section{Organisation générale}

   Comme tout système digne de ce nom, ManuX-32 définit un mode
utilisateur et un mode noyau. Il n'utilise pour l'instant pas les
différents modes fournis par le processeur, mais j'ai bon espoir pour
que cela arrive bientôt.

%-------------------------------------------------------------------------------
%   Structure du code
%-------------------------------------------------------------------------------
\subsection{Structure du code}

   Le code de ManuX est réparti dans quelques fichiers, eux-mêmes
organisés dans quelques répertoires.

   Au plus haut niveau, nous allons distinguer deux grandes parties
   que sont le ``code noyau'' et le ``code utilisateur''.
   
%...............................................................................
%
%...............................................................................
\subsubsection{Le code ``utilisateur''}

   Le code du monde utilisateur est confiné dans le répertoire {\tt
usr}.
  Les principaux fichiers de {\tt usr} sont les suivants

\begin{description}
   \item[{\tt unistd.c}] implante les principales fonctions (décrites
     dans {\tt include/unistd.h}) permettant l'accès aux fonctions du
     noyau. Comme nous le verrons, ce sont donc surtout des appels
     systèmes. 
   \item[{\tt stdio.c}] implante les principales fonctions (décrites
     dans {\tt include/stdio.h}) permettant de réaliser des
     entrées/sorties.
   \item[{\tt init.c}] implante le code de la première tâche lancée
     automatiquement par le noyau lorsqu'il est initialisé.
\end{description}

   La compilation du code utilisateur nécessite naturellement des
fichiers inclus. Ils se trouvent dans le répertoire {\tt
  usr/include}. Certains
     des fichiers, situés dans le répertoire {\tt include/manux} sont
     en fait des copies depuis l'arborescence du noyau. Ils sont mis à
     jour au travers de la cible {\tt usrinc} du {\tt Makefile
       principal}. Il s'agit de 

\begin{description}
   \item[{\tt config.h}] qui décrit la configuration du noyau, il est
     donc important de pouvoir en disposer dans le mode utilisateur
     afin de savoir ce qui est disponible ;
   \item[{\tt types.h}] qui définit les types utilisés dans le noyau ;
   \item[{\tt appelsystemenum.h}] qui donne les numéros des appels
     systèmes ;
   \item[{\tt string.h}] est là pour offir au mode utilisateur des
     fonctions présentes dans le noyau ({\tt memcpy()}, \ldots). Il
     n'est pas nécessaire que ce soit le même, mais cela évite de
     tenir à jour deux fichiers !
\end{description}

%...............................................................................
%
%...............................................................................
\subsubsection{Le code du noyau proprement dit}


\begin{description}
   \item[{\tt lib}]
   \item[{\tt noyau}]
   \item[{\tt i386}]
   \item[{\tt outils}]
   \item[{\tt boot}]
   \item[{\tt sf}]
   \item[{\tt doc}]
\end{description}

%-------------------------------------------------------------------------------
%   
%-------------------------------------------------------------------------------
\subsection{Configuration}

   La configuration se fait essentiellement au travers du fichier {\tt manux/config.h}.

   Certaines valeurs sont utilisées ailleurs que dans du C (dans du NASM,
ou directement dans le {\tt Makefile}) et une cible  {\tt make} est là
pour générer un fichier {\tt make.conf} qui contient ces
variables. Pour qu'elle fonctionne, il est impératif que ces macros
débutent par \lstinline!MANUX\_!

   D'autres valeurs doivent être déterminées durant la phase de
compilation (par exemple la taille du noyau). Elles sont donc
calculées par le programme {\tt outils/taillenoyau} et placées dans le
fichier {\tt taille.conf}.

%...............................................................................
%
%...............................................................................
\subsubsection{Le fichier {\tt include/config.h}}

   Il permet de choisir l'inclusion, ou non, de certines parties du
noyau, et parfois de les paramétrer.

%
%
%
\paragraph{Configuration du fichier de boot}

\begin{description}
   \item[{\tt MANUX\_TAILLE\_PAGE}]
   \item[{\tt MANUX\_NOMBRE\_PAGES\_SYSTEME}]
   \item[{\tt MANUX\_ADRESSE\_DEBUT\_TAS}]
   \item[{\tt MANUX\_NB\_MAX\_FICHIERS}]
   \item[{\tt MANUX\_CONSOLES\_VIRTUELLES}]
   \item[{\tt MANUX\_JOURNAL}]
\end{description}

%
%
%
\paragraph{Configuration de l'adressage du noyau}
\begin{description}
   \item[{\tt MANUX\_TAILLE\_PAGE}]
   \item[{\tt MANUX\_NOMBRE\_PAGES\_SYSTEME}]
   \item[{\tt MANUX\_ADRESSE\_DEBUT\_TAS}]
\end{description}

%
%
%
\paragraph{Configuration de l'affichage}

\begin{description}
   \item[{\tt MANUX\_CONSOLES\_VIRTUELLES}]
   \item[{\tt MANUX\_JOURNAL}]
\end{description}

%
%
%
\paragraph{Configuration du système de fichiers}

\begin{description}
   \item[{\tt NB\_MAX\_FICHIERS}]
\end{description}

%...............................................................................
%
%...............................................................................
\subsubsection{Le fichier {\tt make.conf}}

%...............................................................................
%
%...............................................................................
\subsubsection{Le fichier {\tt taille.conf}}


%-------------------------------------------------------------------------------
%   
%-------------------------------------------------------------------------------
\subsection{Organisation de ce document}

   Mon but est de décrire les choses de façon progressive, dans
l'ordre dans lequel je les ai faites, et dans l'ordre dans lequel il
me semble pertinent de les faire de sorte à construire progressivement
un noyau qui fonctionne.

   La section ``{\em Les outils de développement}'' décrit sommairement
les outils nécessaires à la compilation, le débogage et le lancement
de ManuX.

   La section ``{\em Le boot}'' montre comment faire en sorte que notre
code soit chargé en mémoire par le {\sc bios} de la machine puis
exécuté. C'est une partie assez technique, très spécifique au matériel
ciblé, et peu intéressante. Je vous autorise à la passer. À la fin de
cette partie, le code du noyau lui même commence à être exécuté, c'est
donc là que les choses deviennent croustillantes ! Mais c'est
également là que les ennuis commencent, \ldots

   Afin de pouvoir voir ce que nous faisons, une première activité va
consister à écrire du code pouvant afficher des choses à l'écran, nous
pourrons enfin avoir notre ``{\tt Hello world !}'' du noyau. Pour
cela, nous allons créer une ``console'' et une fonction
\lstinline!printk! dans la section ``{\em L'affichage}''.

   La section ``{\em Les appels systèmes}'' sera ensuite consacrée à
mettre en place le mécanisme permettant aux applications de
communiquer avec le noyau. Pour le moment, ce mécanisme n'est pas
nécessaire, un simple appel de fonction permet cela, mais lorsque nous
mettrons en place des mécanimes de protection, nous seronsbien
contents d'avoir fait les choses proprement !

   Dans la section ``{\em Un premier {\tt init}}'', nous alons écrire
un premier ``code utilisateur''

   Nous allons ensuite nous attaquer à un morceau assez conséquent :
la gestion de la mémoire qui sera décrite dans la section ``{\em La
  gestion de la mémoire}''. Nous ferons dans ManuX les choses aussi
simplement que possible. 
   
